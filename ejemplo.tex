\documentclass[a4paper, 12pt]{article}

\usepackage[utf8]{inputenc}
\usepackage[spanish]{babel}
\usepackage{graphicx}
\usepackage{amssymb}

\topmargin=-1cm
\oddsidemargin=0cm
\textheight=24cm
\textwidth=17cm

\newcommand{\dire}[2]{\footnote{e-mail: {#1}. Institución: {#2}}}

\begin{document}
\renewcommand{\tablename}{Tabla}
\title{{\bfseries Aproximación de funciones por polinomios:}\\
Polinomio de Taylor.}
\author{Carlos Mauricio Silva \dire{carlosmauriciosilva@gmail.com}
{Facultad de Ciencias Exactas, Ingeniería. y Agrimensura}}
\date{\normalsize{Febrero de 2008}}
\maketitle

\begin{abstract}
Se explica como se origina un polinomio de Taylor, y como se obtiene su error.
\end{abstract}

\section{Polinomio de Taylor}
Para aproximar el valor de una función en un punto utilizamos un
polinomio de Taylor.

Estos polinomios se definen como
\begin{equation}
\label{eq:definicion}
T_{n}\left(f,a\right)\!\left(x\right)=
\sum_{k=0}^n\frac{f^{\left(k\right)}\!(a)}{k!}\cdot x^k
\end{equation}
donde $a$ es el punto alrededor del cual engendramos el polinomio
y $n$ es el grado del polinomio.

Esta definición surge de buscar un polinomio que cumpla las $n+1$
condiciones: $$f^{(k)}(a)=T_n^{(k)}(a)\hspace{20mm}k=1,\ldots,n$$
En el caso particular en el que el polinomio se engendra en el punto
$a=0$ obtendremos la fórmula
\begin{equation}
\label{eq:mclaurin}
T_{n}\left(f,0\right)\!\left(x\right)=
\sum_{k=0}^n\frac{f^{\left(k\right)}\!(0)}{k!}\cdot x^k
\end{equation}

\section {Error en la fórmula de Taylor.}
Para saber cuan buena es la aproximación que nos dan estos polinomios
calculamos su error, que notaremos
$R_{n}\left(f,a\right)\!\left(x\right)$.
Para calcular este resto conocemos dos formas:
la forma integral y la forma de Lagrange.

\noindent{Forma Integral del resto:}
\begin{equation}
\label{eq:integral}
R_{n}\left(f,a\right)\!\left(x\right)=
\frac1{n!} \int_a^x\left(x-t\right)f^{\left(n+1\right)}\!(t)\ dt
\end{equation}

\noindent{Forma de Lagrange del Resto:}
\begin{equation}
\label{eq:lagrange}
R_{n}\left(f,a\right)\!\left(x\right)=
\frac1{(n+1)!} f^{\left(n+1\right)}\!(\theta\!\cdot\!(x-a))(x-a)^{n+1}
\end{equation}
donde $\theta$ es un número real entre $0$ y $1$.

Volviendo al caso particular en el que $a=0$ veremos simplificada
la expresión \ref{eq:lagrange}, teniendo ahora la forma:
$$R_{n}\left(f,0\right)\!\left(x\right)=
\frac1{(n+1)!} f^{\left(n+1\right)}\!(\theta x)x^{n+1}$$

\end{document}